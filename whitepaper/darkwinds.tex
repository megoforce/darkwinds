
\title{
        Darkwinds \\
        \large A trading card game on the Ethereum Blockchain
        }
\author{DRAFT\\Cristian Gonzalez, MEGO Corp.}
\date{\today}

\documentclass[11pt,twocolumn]{article}
\usepackage{tikz}
\usepackage{pgfplots}
\usepackage{filecontents}

\begin{filecontents}{datax.dat}
        0,40062
        1,40818
        2,41101
        3,41044
        4,41086
        5,40869
        6,41829
        7,42071
        8,41995
        9,41869
        10,41751
        11,42834
        12,43049
        13,42940
        14,43120
        15,42870
        16,43833
        17,44021
        18,44062
        19,44074
        20,43830
        21,44584
        22,45134
        23,44863
        24,45117
        25,44837
        26,45829
        27,46187
        28,46063
        29,46197
        30,46171
        31,46906
        32,47459
        33,46999
        34,47182
        35,46952
        36,47947
        37,48331
        38,48152
        39,48122
        40,47973
        41,48936
        42,49403
        43,48959
        44,49142
        45,48829
        46,49834
        47,50184
        48,49997
        49,49763
        50,9908
        51,8989
        52,9306
        53,8808
        54,9136
        55,8918
        56,8071
        57,8113
        58,8052
        59,7774
        60,7904
        61,6946
        62,7266
        63,6868
        64,7117
        65,6929
        66,6073
        67,6125
        68,5983
        69,5848
        70,5963
        71,4942
        72,5175
        73,4947
        74,5106
        75,5012
        76,3964
        77,4003
        78,3966
        79,3940
        80,3993
        81,2964
        82,3086
        83,2897
        84,3079
        85,3040
        86,2024
        87,1990
        88,1993
        89,1989
        90,2015
        91,1004
        92,1051
        93,924
        94,1010
        95,1027
    \end{filecontents}
    
\begin{document}
\maketitle
\pagebreak

\pagebreak
\twocolumn
\section{Introduction}
\begin{quote}"I happily played World of Warcraft during 2007-2010, but one day Blizzard removed the damage component from my beloved warlock's Siphon Life spell. I cried myself to sleep, and on that day I realized what horrors centralized services can bring."
        \begin{flushright}
                {---Vitalik Buterin, Creator of Ethereum}
              \end{flushright}
        \end{quote}

Collectible card games have existed in popular demand since decades. After the massification of the Internet, many of them were successfully digitized like \textit{Magic: The Gathering} and \textit{Pokemon} and new ones were created emulating the look and feel of the physical collectibles in form of a multiplayer videogame.
However, being of electronic nature, it's physical existence is an entry on a database, and the amount of each card each player owns is a counter inside a central server. Using databases to create trading card games can produce a fun experience at the cost of conceptual problems:

\begin{itemize}
        \item Only the owner of the software can allow/deny the use of the digital card within it's ecosystem.
        \item The digital card existence depends on the game systems to be online.
        \item The game owner can prohibit trading or markets for cards, which is usually enforced.
        \item The game owner can control the issuance of cards witout transparency to the user.
\end{itemize}

Thanks to Blockchain technology we can make a trading card game where all cards are stored in a Ethereum smart contract, where players obtain true ownership of the game object and are free to trade in an outside the game program.\\

By being a blockchain asset, each card will have the following qualities:

\begin{itemize}
        \item It's securely stored in an Ethereum wallet.
        \item The owner and only the owner has perpetual rights for transfer, trade or sale of the token
        \item It can be read by other applications, like a mobile wallet for ERC721 tokens.
        \item It's existence doesn't depend on the availability of the game owner systems.
\end{itemize}

Each card is composed of a serial number and a model id that corresponds to a card model. There can be many cards of the same model, with different serial numbers. Card models are contain metadata for it's appearance (an image file) and it's game action properties (ex: deals 3 damage points to enemy target).\\

The smart contract also controls immutable rules on the issuance of cards, including the total limit of cards to be issued and the rarity of card models.


\section{The Game}
Darkwinds is an online trading card game where two players confront each other in a sea battle of ships.\\

While the smart contract handles the ownership of tokens, game matches occur off-chain, in a webGL website running the Metamask web extension or compatible thin wallets. A game server is responsible for matchmaking between two adversaries, validating the signature of both players, thus verfying the ownership of both player decks. While the official game server will be the only endorsed way of playing Darkwinds, other developers are free to read the ABIs and access players cards to create different game modes, tournaments or applications that connect to the game.\\

Game servers only require a signed message from the user wallet to verify ownership. The user private keys are never read or stored.


\section{Anatomy of a Darkwinds token}
The card smart contract performs all the functions according to the ERC721 \cite{notes} standard. The cards are generated with a payable function called getBoosterPack and the amount of cards returned to the owner is determined by the price of cards, which is set by the owner using the setCardPrice function
\subsection{getCard}
Gets all the metadata for the card relevant to the game mechanics (actions, cost of invocation, effects, etc.)
\subsection{getBoosterPack}
Spends a determined amount of ETH to get a booster pack. Normally it will contain 10, 25 or 50 cards but the contract can be excecuted for a specific amount of cards multiplying the desired amount of cards with the cost.
\subsection{changeCardCost}
A function designed for the owner to change the price of the card, and by consequence the price of booster packs. By default is 1 finney.
\section{Card Generation}
Cards are generated with a KECCAK-256 operation on the last block timestamp,  that result in a limited distribution on certain rare cards. However, since the block timestamp is only incrementing it's possible to approximate the final distribution of cards.

While it's possible to determine exactly when cards are being released, efforts are probably not practical. 

The smart contract stops generating cards when the hard cap of 1,000,000 is reached.
\begin{figure}[!htb]
\centering 
    
\begin{tikzpicture}
\begin{axis}[xlabel={Card Model},  scaled y ticks=base 10:-3,
        ytick scale label code/.code={},
        yticklabel={\pgfmathprintnumber{\tick}.000}]

% Graph column 2 versus column 0
\addplot table[x index=0,y index=1,col sep=comma] {datax.dat};


\end{axis}
\end{tikzpicture}
\caption{Distribution of 1 million cards estimated by a Monte Carlo simulation} \label{fig:A}
\end{figure}


\onecolumn

\begin{thebibliography}{1}

        \bibitem{notes} The ERC721 non-fungible token standard. https://github.com/ethereum/eips/issues/721, Dieter Shirley, 2017.
      
        \end{thebibliography}
\end{document}